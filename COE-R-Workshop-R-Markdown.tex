\PassOptionsToPackage{unicode=true}{hyperref} % options for packages loaded elsewhere
\PassOptionsToPackage{hyphens}{url}
%
\documentclass[
]{article}
\usepackage{lmodern}
\usepackage{amssymb,amsmath}
\usepackage{ifxetex,ifluatex}
\ifnum 0\ifxetex 1\fi\ifluatex 1\fi=0 % if pdftex
  \usepackage[T1]{fontenc}
  \usepackage[utf8]{inputenc}
  \usepackage{textcomp} % provides euro and other symbols
\else % if luatex or xelatex
  \usepackage{unicode-math}
  \defaultfontfeatures{Scale=MatchLowercase}
  \defaultfontfeatures[\rmfamily]{Ligatures=TeX,Scale=1}
\fi
% use upquote if available, for straight quotes in verbatim environments
\IfFileExists{upquote.sty}{\usepackage{upquote}}{}
\IfFileExists{microtype.sty}{% use microtype if available
  \usepackage[]{microtype}
  \UseMicrotypeSet[protrusion]{basicmath} % disable protrusion for tt fonts
}{}
\makeatletter
\@ifundefined{KOMAClassName}{% if non-KOMA class
  \IfFileExists{parskip.sty}{%
    \usepackage{parskip}
  }{% else
    \setlength{\parindent}{0pt}
    \setlength{\parskip}{6pt plus 2pt minus 1pt}}
}{% if KOMA class
  \KOMAoptions{parskip=half}}
\makeatother
\usepackage{xcolor}
\IfFileExists{xurl.sty}{\usepackage{xurl}}{} % add URL line breaks if available
\IfFileExists{bookmark.sty}{\usepackage{bookmark}}{\usepackage{hyperref}}
\hypersetup{
  pdftitle={COE Intro to R Workshop},
  pdfauthor={Maria Schweer-Collins, PhD \& Camille Cioffi, PhD},
  pdfborder={0 0 0},
  breaklinks=true}
\urlstyle{same}  % don't use monospace font for urls
\usepackage[margin=1in]{geometry}
\usepackage{color}
\usepackage{fancyvrb}
\newcommand{\VerbBar}{|}
\newcommand{\VERB}{\Verb[commandchars=\\\{\}]}
\DefineVerbatimEnvironment{Highlighting}{Verbatim}{commandchars=\\\{\}}
% Add ',fontsize=\small' for more characters per line
\usepackage{framed}
\definecolor{shadecolor}{RGB}{248,248,248}
\newenvironment{Shaded}{\begin{snugshade}}{\end{snugshade}}
\newcommand{\AlertTok}[1]{\textcolor[rgb]{0.94,0.16,0.16}{#1}}
\newcommand{\AnnotationTok}[1]{\textcolor[rgb]{0.56,0.35,0.01}{\textbf{\textit{#1}}}}
\newcommand{\AttributeTok}[1]{\textcolor[rgb]{0.77,0.63,0.00}{#1}}
\newcommand{\BaseNTok}[1]{\textcolor[rgb]{0.00,0.00,0.81}{#1}}
\newcommand{\BuiltInTok}[1]{#1}
\newcommand{\CharTok}[1]{\textcolor[rgb]{0.31,0.60,0.02}{#1}}
\newcommand{\CommentTok}[1]{\textcolor[rgb]{0.56,0.35,0.01}{\textit{#1}}}
\newcommand{\CommentVarTok}[1]{\textcolor[rgb]{0.56,0.35,0.01}{\textbf{\textit{#1}}}}
\newcommand{\ConstantTok}[1]{\textcolor[rgb]{0.00,0.00,0.00}{#1}}
\newcommand{\ControlFlowTok}[1]{\textcolor[rgb]{0.13,0.29,0.53}{\textbf{#1}}}
\newcommand{\DataTypeTok}[1]{\textcolor[rgb]{0.13,0.29,0.53}{#1}}
\newcommand{\DecValTok}[1]{\textcolor[rgb]{0.00,0.00,0.81}{#1}}
\newcommand{\DocumentationTok}[1]{\textcolor[rgb]{0.56,0.35,0.01}{\textbf{\textit{#1}}}}
\newcommand{\ErrorTok}[1]{\textcolor[rgb]{0.64,0.00,0.00}{\textbf{#1}}}
\newcommand{\ExtensionTok}[1]{#1}
\newcommand{\FloatTok}[1]{\textcolor[rgb]{0.00,0.00,0.81}{#1}}
\newcommand{\FunctionTok}[1]{\textcolor[rgb]{0.00,0.00,0.00}{#1}}
\newcommand{\ImportTok}[1]{#1}
\newcommand{\InformationTok}[1]{\textcolor[rgb]{0.56,0.35,0.01}{\textbf{\textit{#1}}}}
\newcommand{\KeywordTok}[1]{\textcolor[rgb]{0.13,0.29,0.53}{\textbf{#1}}}
\newcommand{\NormalTok}[1]{#1}
\newcommand{\OperatorTok}[1]{\textcolor[rgb]{0.81,0.36,0.00}{\textbf{#1}}}
\newcommand{\OtherTok}[1]{\textcolor[rgb]{0.56,0.35,0.01}{#1}}
\newcommand{\PreprocessorTok}[1]{\textcolor[rgb]{0.56,0.35,0.01}{\textit{#1}}}
\newcommand{\RegionMarkerTok}[1]{#1}
\newcommand{\SpecialCharTok}[1]{\textcolor[rgb]{0.00,0.00,0.00}{#1}}
\newcommand{\SpecialStringTok}[1]{\textcolor[rgb]{0.31,0.60,0.02}{#1}}
\newcommand{\StringTok}[1]{\textcolor[rgb]{0.31,0.60,0.02}{#1}}
\newcommand{\VariableTok}[1]{\textcolor[rgb]{0.00,0.00,0.00}{#1}}
\newcommand{\VerbatimStringTok}[1]{\textcolor[rgb]{0.31,0.60,0.02}{#1}}
\newcommand{\WarningTok}[1]{\textcolor[rgb]{0.56,0.35,0.01}{\textbf{\textit{#1}}}}
\usepackage{longtable,booktabs}
% Allow footnotes in longtable head/foot
\IfFileExists{footnotehyper.sty}{\usepackage{footnotehyper}}{\usepackage{footnote}}
\makesavenoteenv{longtable}
\usepackage{graphicx,grffile}
\makeatletter
\def\maxwidth{\ifdim\Gin@nat@width>\linewidth\linewidth\else\Gin@nat@width\fi}
\def\maxheight{\ifdim\Gin@nat@height>\textheight\textheight\else\Gin@nat@height\fi}
\makeatother
% Scale images if necessary, so that they will not overflow the page
% margins by default, and it is still possible to overwrite the defaults
% using explicit options in \includegraphics[width, height, ...]{}
\setkeys{Gin}{width=\maxwidth,height=\maxheight,keepaspectratio}
\setlength{\emergencystretch}{3em}  % prevent overfull lines
\providecommand{\tightlist}{%
  \setlength{\itemsep}{0pt}\setlength{\parskip}{0pt}}
\setcounter{secnumdepth}{-2}
% Redefines (sub)paragraphs to behave more like sections
\ifx\paragraph\undefined\else
  \let\oldparagraph\paragraph
  \renewcommand{\paragraph}[1]{\oldparagraph{#1}\mbox{}}
\fi
\ifx\subparagraph\undefined\else
  \let\oldsubparagraph\subparagraph
  \renewcommand{\subparagraph}[1]{\oldsubparagraph{#1}\mbox{}}
\fi

% set default figure placement to htbp
\makeatletter
\def\fps@figure{htbp}
\makeatother


\title{COE Intro to R Workshop}
\author{Maria Schweer-Collins, PhD \& Camille Cioffi, PhD}
\date{Fall 2020}

\begin{document}
\maketitle

\hypertarget{introduction-to-r-packages}{%
\subsection{Introduction to R
Packages}\label{introduction-to-r-packages}}

\hypertarget{installing-and-loading-r-packages}{%
\subsubsection{Installing and Loading R
Packages}\label{installing-and-loading-r-packages}}

\begin{Shaded}
\begin{Highlighting}[]
\CommentTok{# how to install R packages.}
\CommentTok{# do this the first time you use the package}
\KeywordTok{install.packages}\NormalTok{(}\StringTok{"tidyverse"}\NormalTok{)}
\CommentTok{# each time you open a new R session, you have to load the package}
\KeywordTok{library}\NormalTok{(tidyverse)}

\CommentTok{# check for a package update}
\KeywordTok{update.packages}\NormalTok{(}\StringTok{"tidyverse"}\NormalTok{)}
\end{Highlighting}
\end{Shaded}

\hypertarget{set-working-directory}{%
\subsubsection{Set Working Directory}\label{set-working-directory}}

This will tell you where your files will be saved

\begin{Shaded}
\begin{Highlighting}[]
\CommentTok{# to set your working directory you can use setwd("C:/FILE PATH")}

\CommentTok{# Example}
\CommentTok{# setwd("C:/Users/jah2ax/Box Sync/_R/workshops/intro_to_R_spring_2019")}
\end{Highlighting}
\end{Shaded}

\hypertarget{import-data-into-r}{%
\subsubsection{Import Data into R}\label{import-data-into-r}}

\begin{Shaded}
\begin{Highlighting}[]
\CommentTok{# example od importing a .csv SPSS file}

\NormalTok{data1 <-}\StringTok{ }\KeywordTok{read.csv}\NormalTok{(}\StringTok{"data.csv"}\NormalTok{)}


\CommentTok{# example of importing a .sav file using package Haven}
\CommentTok{# you can also import SAS and Stata files with Haven}
\CommentTok{# install.packages("haven")}
\KeywordTok{library}\NormalTok{(haven)}
\NormalTok{data2 <-}\StringTok{ }\KeywordTok{read_sav}\NormalTok{(}\StringTok{"data.sav"}\NormalTok{)}
\end{Highlighting}
\end{Shaded}

\hypertarget{viewing-your-data-descriptive-statistics}{%
\subsubsection{Viewing your Data / Descriptive
Statistics}\label{viewing-your-data-descriptive-statistics}}

\begin{Shaded}
\begin{Highlighting}[]
\CommentTok{# use package "skimr" to get the mean, standard deviation, histogram, and percentiles of numeric data}
\CommentTok{# install.packages("skimr")}
\KeywordTok{library}\NormalTok{(skimr)}

\CommentTok{# use skim function to review data}
\KeywordTok{skim}\NormalTok{(data1)}
\end{Highlighting}
\end{Shaded}

\begin{longtable}[]{@{}ll@{}}
\caption{Data summary}\tabularnewline
\toprule
\endhead
Name & data1\tabularnewline
Number of rows & 32\tabularnewline
Number of columns & 11\tabularnewline
\_\_\_\_\_\_\_\_\_\_\_\_\_\_\_\_\_\_\_\_\_\_\_ &\tabularnewline
Column type frequency: &\tabularnewline
numeric & 11\tabularnewline
\_\_\_\_\_\_\_\_\_\_\_\_\_\_\_\_\_\_\_\_\_\_\_\_ &\tabularnewline
Group variables & None\tabularnewline
\bottomrule
\end{longtable}

\textbf{Variable type: numeric}

\begin{longtable}[]{@{}lrrrrrrrrrl@{}}
\toprule
skim\_variable & n\_missing & complete\_rate & mean & sd & p0 & p25 &
p50 & p75 & p100 & hist\tabularnewline
\midrule
\endhead
mpg & 0 & 1 & 20.09 & 6.03 & 10.40 & 15.43 & 19.20 & 22.80 & 33.90 &
▃▇▅▁▂\tabularnewline
cyl & 0 & 1 & 6.19 & 1.79 & 4.00 & 4.00 & 6.00 & 8.00 & 8.00 &
▆▁▃▁▇\tabularnewline
disp & 0 & 1 & 230.72 & 123.94 & 71.10 & 120.83 & 196.30 & 326.00 &
472.00 & ▇▃▃▃▂\tabularnewline
hp & 0 & 1 & 146.69 & 68.56 & 52.00 & 96.50 & 123.00 & 180.00 & 335.00 &
▇▇▆▃▁\tabularnewline
drat & 0 & 1 & 3.60 & 0.53 & 2.76 & 3.08 & 3.70 & 3.92 & 4.93 &
▇▃▇▅▁\tabularnewline
wt & 0 & 1 & 3.22 & 0.98 & 1.51 & 2.58 & 3.33 & 3.61 & 5.42 &
▃▃▇▁▂\tabularnewline
qsec & 0 & 1 & 17.85 & 1.79 & 14.50 & 16.89 & 17.71 & 18.90 & 22.90 &
▃▇▇▂▁\tabularnewline
vs & 0 & 1 & 0.44 & 0.50 & 0.00 & 0.00 & 0.00 & 1.00 & 1.00 &
▇▁▁▁▆\tabularnewline
am & 0 & 1 & 0.41 & 0.50 & 0.00 & 0.00 & 0.00 & 1.00 & 1.00 &
▇▁▁▁▆\tabularnewline
gear & 0 & 1 & 3.69 & 0.74 & 3.00 & 3.00 & 4.00 & 4.00 & 5.00 &
▇▁▆▁▂\tabularnewline
carb & 0 & 1 & 2.81 & 1.62 & 1.00 & 2.00 & 2.00 & 4.00 & 8.00 &
▇▂▅▁▁\tabularnewline
\bottomrule
\end{longtable}

\begin{Shaded}
\begin{Highlighting}[]
\CommentTok{# use the package "psych" to get the mean, SD, range, min, max, median, skewness, and kurtosis of variables in dataset}

\CommentTok{# install.packages("psych")}
\KeywordTok{library}\NormalTok{(psych)}

\CommentTok{#use describe function to review data}
\KeywordTok{describe}\NormalTok{(data1)}
\end{Highlighting}
\end{Shaded}

\begin{verbatim}
##      vars  n   mean     sd median trimmed    mad   min    max  range  skew
## mpg     1 32  20.09   6.03  19.20   19.70   5.41 10.40  33.90  23.50  0.61
## cyl     2 32   6.19   1.79   6.00    6.23   2.97  4.00   8.00   4.00 -0.17
## disp    3 32 230.72 123.94 196.30  222.52 140.48 71.10 472.00 400.90  0.38
## hp      4 32 146.69  68.56 123.00  141.19  77.10 52.00 335.00 283.00  0.73
## drat    5 32   3.60   0.53   3.70    3.58   0.70  2.76   4.93   2.17  0.27
## wt      6 32   3.22   0.98   3.33    3.15   0.77  1.51   5.42   3.91  0.42
## qsec    7 32  17.85   1.79  17.71   17.83   1.42 14.50  22.90   8.40  0.37
## vs      8 32   0.44   0.50   0.00    0.42   0.00  0.00   1.00   1.00  0.24
## am      9 32   0.41   0.50   0.00    0.38   0.00  0.00   1.00   1.00  0.36
## gear   10 32   3.69   0.74   4.00    3.62   1.48  3.00   5.00   2.00  0.53
## carb   11 32   2.81   1.62   2.00    2.65   1.48  1.00   8.00   7.00  1.05
##      kurtosis    se
## mpg     -0.37  1.07
## cyl     -1.76  0.32
## disp    -1.21 21.91
## hp      -0.14 12.12
## drat    -0.71  0.09
## wt      -0.02  0.17
## qsec     0.34  0.32
## vs      -2.00  0.09
## am      -1.92  0.09
## gear    -1.07  0.13
## carb     1.26  0.29
\end{verbatim}

\hypertarget{useful-r-packages}{%
\subsubsection{Useful R Packages}\label{useful-r-packages}}

\begin{Shaded}
\begin{Highlighting}[]
\CommentTok{# data vizualization}
\KeywordTok{library}\NormalTok{(tidyverse)}
\CommentTok{# APA Tables}
\KeywordTok{library}\NormalTok{(apaTables)}
\KeywordTok{library}\NormalTok{(sjPlot)}
\end{Highlighting}
\end{Shaded}

\begin{enumerate}
\def\labelenumi{\alph{enumi}.}
\item
  Saving your data in the same folder as your project
\item
  Tibble and dataframes: uses and differences
\end{enumerate}

\url{https://r4ds.had.co.nz/tibbles.html}

\begin{enumerate}
\def\labelenumi{\alph{enumi}.}
\setcounter{enumi}{2}
\tightlist
\item
  Tools to preview data, and summary statistics
\end{enumerate}

\url{https://learn.datacamp.com/courses/introduction-to-the-tidyverse}

\begin{enumerate}
\def\labelenumi{\alph{enumi}.}
\setcounter{enumi}{3}
\tightlist
\item
  Hot keys:
  \url{https://support.rstudio.com/hc/en-us/articles/200711853-Keyboard-Shortcuts}
\end{enumerate}

G. Managing Data (For Advanced R users to move ahead)

\url{https://learn.datacamp.com/courses/data-manipulation-with-dplyr}

\end{document}
